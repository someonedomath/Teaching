% * Preamble
% ** Document class
\documentclass[12pt,a4paper]{amsart}
\setlength{\textwidth}{\paperwidth}
\addtolength{\textwidth}{-2in}
\calclayout
% ** Encoding information
\usepackage[utf8]{inputenc}
\usepackage[T1]{fontenc}
% \usepackage[UTF8]{ctex} 
% You need XeTeX to build ctex based document 
% ** Packages about math
\usepackage{mathtools}
\mathtoolsset{showonlyrefs}
\usepackage{mathrsfs}
\usepackage{stackrel}
\usepackage[backref]{hyperref}
\usepackage{comment}
% ** Theorem environment definition
\usepackage{amsthm}
\theoremstyle{plain}
\newtheorem{thm}{Theorem}[section]
\newtheorem{lem}[thm]{Lemma}
\newtheorem{prop}[thm]{Proposition}
\newtheorem{cor}[thm]{Corollaray}
\newtheorem{conj}[thm]{Conjecture}
\theoremstyle{definition}
\newtheorem{defi}[thm]{Definition}
\newtheorem{exa}[thm]{Example}
\newtheorem{asp}{Assumption}
\theoremstyle{remark}
\newtheorem*{exe}{Exercise}
\newtheorem{rem}[thm]{Remark}
\numberwithin{equation}{section}
\allowdisplaybreaks
% * Document
% ** BEGIN
\begin{document}
% ** Basic Info
\title
[Solutions]
{Solutions to the Selected Exercises in R. Durrett's Probability: Theory and Examples, II}
\author
[Z. Sun]
{Zhenyao Sun}
\address
{Zhenyao Sun\\
School of Mathematical Sciences\\
Peking University\\
Beijing, P. R. China, 100871}
\email{zhenyao.sun@pku.edu.cn}
%\begin{abstract}
%  TBD
%\end{abstract}
\maketitle
% ** Section: HW for 6.5
% *** 6.5.1.
\begin{exe}[6.5.1]
To show that the convergence in (a) of Theorem 6.4.1. may occur arbitrarily slowly, let $X_{m,m+k}=f(k)\geq 0$, where $f(k)/k$ is decreasing, and check that $X_{m,m+k}$ is subaddditive.
\end{exe}
\begin{proof}
Verify (i): 
\begin{align}
X_{0,m}+X_{m,n} 
&= f(m) + f(n-m)
= m \frac{f(m)}{m} + (n-m)  \frac{f(n-m)}{n-m} 
\\&\geq m \frac{f(n)}{n} + (n-m) \frac{f(n)}{n}
= f(n) 
= X_{0,n}.
\end{align}

Verify (ii): For each $k$, $(X_{nk,(n+1)k})_{n\geq 1}= (f(k))_{n\geq 1}$ is obviously a stationary sequence.

Verify (iii): The distribution of $(X_{m,m+k})_{k\geq 1} = (f(k))_{k\geq 1}$ obviously does not depend on $m$.

Verify (iv): Obviously $EX_{0,1}^+=f(1)<\infty$. 
Denote by $\gamma_0 := \lim_{k\to \infty} f(k)/k \geq 0$, then we do have $EX_{0,n}= f(n)\geq \gamma_0 n$ and $\gamma_0 > -\infty$. 
\end{proof}
% *** 6.5.2.
\begin{exe}[6.5.2.]
Consider the longest common subsequence problem, Example 6.4.4. when $X_1,X_2,\dots$ and $Y_1,Y_2,\dots$ are i.i.d. and take the values $0$ and $1$ with probability $1/2$ each.
(a) Compute $EL_1$ and $EL_2/2$ to get lower bounds on $\gamma$.
(b) Show $\gamma < 1$ by computing the expected number of $i$ and $j$ sequence of length $K=an$ with the desired property. 
\end{exe}
\begin{proof}
(a) Since $L_{0,1} \in [0,1] \cap \mathbb Z$, we have
\begin{align}
E L_{0,1} 
= P(L_{0,1} = 1) 
= P(X_1 = Y_1)
= 1/2.
\end{align}
Similarly, since $L_{0,2} \in [0,2]\cap \mathbb Z$, we have
\begin{align}
E L_{0,2}
& = P(L_{0,2} = 1) + 2P(L_{0,2} = 2)
= \left( 1- P(L_{0,2} = 0) - P(L_{0,2} = 2) \right) + 2 P(L_{0,2} = 2)
\\&= 1 + P(L_{0,2} = 2) - P(L_{0,2} = 0)
= 1+ P(X_1=Y_1, X_2= Y_2) - P(X_1 = X_2 \neq Y_1 = Y_2)
\\& = 1 + 1/4 - 1/8 
= 9/8
\end{align} 
So we already know that 
\begin{align}
\gamma = \sup_{m\geq 1} E(L_{0,m})/m \geq 9/16.
\end{align}

(b) For each $k,n \in \mathbb N$ with $k\leq n$, denote by $\mathcal I_{n,k}:= \{ (i_1,\dots, i_k): 1< i_1<\dots < i_k \leq n \}$ the collection of all increasing multi-index with length $k$ in the index space $\{1,2,\dots,n\}$.
For each $I = (i_1,\dots, i_k) \in \mathcal I_{n,k}$, denote by
$X_I:= (X_{i_1},X_{i_2},\dots, X_{i_k})$ the $I$-subsequence of the process $(X_k)_{k\in \mathbb N}$.
Similarly, we can define $Y_I$ for each $I\in \mathcal I_{n,k}$. 
Note that there exists a constant $C>0$ such that for each integers $0 < k < n$, we have
\begin{align}
P\left( L_{0,n} \geq k \right)
&= P(\exists I,J\in \mathcal I_{n,k} ~\text{s.t.}~ X_I = Y_J)
\leq \sum_{I,J\in \mathcal I_{n,k}} P(X_I =Y_J)
\\&= \# \{I,J\in \mathcal I_{n,k}\}\cdot 2^{-k}
=\left(\frac{n!}{k!(n-k)!}\right)^2  2^{-k}
\\&\leq C \left( \frac{n^{n+1/2}e^{-n}}{k^{k+1/2}e^{-k} (n-k)^{n-k+1/2} e^{-{n-k}}} 2^{-k}\right)^2,\quad \text{by Stirling's formula}
\\& \overset{a:=n/k}{=} C \left( \frac{n^{n+1/2}}{(an)^{an+1/2} ((1-a)n)^{(1-a)n+1/2}} 2^{-na} \right)^2 
%! \\& = C \left( \frac{1}{a^{an+1/2} n^{1/2} (1-a)^{(1-a)n+1/2}} 2^{-na} \right)^2 
%! \\& = C n^{-1}\exp \left( -2 (an+1/2)\ln a - 2((1-a)n +1/2)\ln (1-a) - 2na\ln 2 \right)
%! \\& = C n^{-1}\exp \left( -2 n a\ln a  -\ln a - 2(1-a)n \ln(1-a) -\ln (1-a) - 2na\ln 2 \right)
%! \\& = C n^{-1}\exp \left( -2 n (a\ln a+(1-a)\ln (1-a) + a \ln 2)  -\ln a(1-a) \right)
\\& = C \frac{1}{a(1-a)n}\exp \left( -2 n\ln \left ( a^a (1-a)^{1-a} 2^a\right)  \right).
\end{align}
Denote by $g(a) = a^a(1-a)^{1-a}2^a$ for each $a \in (0,1)$, then it holds that $g(a) \xrightarrow[a\uparrow 1]{} 2$, which says that there exists an $0<a_0 < 1$ such that $g(a_0) > 1$. 
Now taking $k_n = \lfloor a_0 n \rfloor$, according to $a_n:=k_n/n \to a_0$, we have
\begin{align}
\sum_{n \in \mathbb N} P\left( \frac{L_{0,n}}{n} \geq a_0 \right)
\leq \sum_{n\in \mathbb N} P\left(L_{0,n} \geq k_n \right)
\leq  C \sum_{n\in \mathbb N} \frac{1}{a_n(1-a_n)n}\exp \left( -2n \ln g(a_n) \right) < \infty.
\end{align}
Therefore, B-C lemma says that almost surely
\begin{align}
\gamma = \lim_{n\to \infty} \frac{L_{0,n}}{n} \leq a_0 < 1.
\end{align}
\end{proof}
% *** 6.5.3
\begin{exe}[6.5.3.]
Given a rate one Poisson process in $[0,\infty)\times [0,\infty)$, let $X_1,Y_1$ be the point that minimizes $x+y$.
Let $(X_2,Y_2)$ be the point in $[X_1,\infty) \times [Y_1,\infty)$ that minimizes $x+y$, and so on.
Use this construction to show that in Example 6.5.2. $\gamma \geq (8/\pi)^{1/2} > 1.59$. 
\end{exe} 
\begin{proof}
The definition of rate one Poisson point process $N$ is given by Example 3.7.7.. 
More precisely, $N(\omega,A)$ is a random measure on $[0,\infty)^2$ such that
\begin{itemize}
\item
For each $w\in \Omega$, $N(w,\cdot)$ is a $\mathbb N\cap \{\infty\}$-valued measure on $[0,\infty)^2$.
\item
For each Borel subset $A\subset [0,\infty)^2$, $N(\cdot, A)$ is a Poisson distributed random variable with mean $\mu(A)$, the Lebesgue measure of $A$.
\end{itemize}
Note that $N$ is a (random) atomic measure, therefore, it is valid to talk about the points of $N$.
According to its definition, $(X_1,Y_1)$ is a point of the atomic measure $N$ in $[0,\infty)^2$ which minimizes $x+y$. 
And $(X_2,Y_2)$ is the point in $[X_1,\infty) \times [Y_1,\infty)$ which minimizes $x+y$. 

We claim the following fact:
$\left(  (X_{k+1}, Y_{k+1}) - (X_k, Y_k)\right)_{k \in \mathbb N}$ are i.i.d. random variables with the same distribution of $(X_1,Y_1)$.
This fact is crucial. Its proof relies on the strong Markov property of the Poisson point processes. Here we omit the details.

Note that for each $t\geq 0$, we have
\begin{align}
&P(X_1+Y_1 > t) 
= P \left( N\left\{(x,y): x\geq 0, y\geq 0, x+y\leq t \right\} = 0 \right)
\\& = e^{- \int_{(x,y): x\geq 0, y\geq 0, x+y \leq t} \mu(dx,dy)}
= e^{-\frac{t^2}{2}}.
\end{align}
Therefore, 
\begin{align}
E[X_1+Y_1] 
= \int_0^\infty P(X_1 + Y_1 > t) dt 
= \sqrt{\frac{\pi}{2}},
\end{align}
which, thanks to the symmetry, says that $EX_1 = EY_1 = \sqrt{\frac{\pi}{8}}$.
Now, from Law of large numbers, we have almost surely
\begin{align}
\frac{Z_n}{n}:= \frac{\max(X_n,Y_n)}{n} \xrightarrow[n \to \infty]{} \sqrt{\frac{\pi}{8}}. 
\end{align}
On the other hand, denoted by $L_{0,s}$ length of the longest increasing path lying in the square $R_{0,n}$ with vertices $(0,0)$, $(0,s)$, $(s,0)$ and $(s,s)$.
(Here the length of a path is simply the number of the 'Poisson' points on that path. So the length of a path is always an integer number.)  
It is now obvious that $(X_1,Y_2),\dots, (X_n,Y_n)$ forms an increasing path in the square $R_{0, Z_n}$. 
Therefore, we have $L_{0,Z_n} \geq n$.
Using the result in Example 6.5.2., we have
\begin{align}
\gamma 
= \lim_{n\to \infty} \frac{L_{0, Z_n}}{ Z_n}  
\geq \lim_{n\to \infty} \frac{n}{ Z_n}
= \sqrt{\frac{8}{\pi}}.
\end{align}
\end{proof}
% *** 6.5.4.
\begin{exe}[6.5.4.]
Let $\pi_n$ be a random permutation of $\{1,\dots,n\}$ and let $J_k^n$ be the number of subsets of $\{1,\dots,n\}$ of size $k$ so that the associated $\pi_n(j)$ form an increasing subsequence.
Compute $EJ_k^n$ and take $k\sim \alpha n^{1/2}$ to conclude that in Example 6.5.2. $\gamma \leq e$.
\end{exe}
\begin{proof}
For each $k,n \in \mathbb N$ with $k\leq n$, denote by $\mathcal H_{n,k}$ the collection of all subset of $\{1,\dots, n\}$ with length $k$. For each $h \in \mathcal H_{n,k}$, we write $h = \{h_i: i=1,\dots, k\}$ such that $0< h_1< \dots < h_k\leq n$. 
There exists a constant $C>0$ such that for each $0<k< n$, we have
\begin{align}
&E J_k^n 
= \sum_{h \in \mathcal H_{n,k}} P\left( \left(\pi_n(h_i)\right)_{i=1}^k~\text{is increasing} \right)
= \# \mathcal H_{n,k} \cdot \frac{ C_n^k \cdot (n-k)!}{n!}
\\&= \frac{n!}{k!(n-k)!k!}
\leq \frac{n^k}{(k!)^2}
%  \leq C \frac{n^{n+\frac{1}{2}}e^{-n}}{ k^{k+\frac{1}{2}} e^{-k} (n-k)^{n-k+\frac{1}{2}}e^{-(n-k)} k^{k+\frac{1}{2}} e^{-k} },
 \leq C \left(\frac{e\sqrt n}{ k } \right )^{2k},
\quad \text{by Stirling formula}. 
\end{align}
Therefore if $k \sim \alpha n^{1/2}$ with $\alpha > e$, we have
\begin{align}
 \sum_{n\in \mathbb N}E J_k^n < \infty.
\end{align}
Now let $l(\pi_n)$ be the length of the longest increasing sequence in the random permutation $\pi_n$, then
\begin{align}
  & \sum_{n\in \mathbb N} P\left( \frac{l(\pi_n)}{n^{1/2}} \geq  \alpha \right)
  =\sum_{n\in \mathbb N} P\left( J_{\left\lceil \alpha n^{1/2}\right\rceil}^n \geq 1\right)
  \\& \leq \sum_{n\in \mathbb N} E J_{\left\lceil \alpha n^{1/2}\right\rceil}^n
  < \infty 
\end{align}
This, and B-C lemma says that
\begin{align}
\gamma 
:= \lim_{n\to \infty} \frac{l(\pi_n)}{n^{1/2}}
\leq \alpha,
\quad \text{almost surely}.
\end{align}
Finally, since $\alpha$ is chosen arbitrarily in $(e,\infty)$, we have that $\gamma \leq e$ almost surely.
\end{proof}
% *** 6.5.5.
\begin{exe}[6.5.5.]
Let $\phi(\theta) = E \exp(-\theta t_i)$ and
\begin{align}
Y_n =\left( \mu \phi(\theta) \right)^{-n} \sum_{i=1}^{Z_n} \exp\left( -\theta T_n(i) \right)
\end{align}
where the sum is over individuals in generation $n$ and $T_n(i)$ is the $i$th person's birth time.
Show that $Y_n$ is a nonnegative martingale and use this to conclude that if $\exp(-\theta a)/\mu \phi(\theta) > 1$, then $P(X_{0,n}\leq an)\to 0$. 
A little thought reveals that this bound is the same as the answer in the answer in the last exercise.
\end{exe}
\begin{proof}
Let $\mathcal F_n$ be the filtration which contains all the information about the birth times of all the persons whose generations are smaller than or equal to $n$.
Denote by $Z_1^{(n,i)}$ the number of children of $i$-th particle in generation $n$. 
Denote by $T_1^{(n,i)}(k)$ the birth time of the $k$-th child of the $i$-th particle in generation $n$.
It can be verified from the independence of the birth of each particles that
\begin{align}
  &E\left[ Y_{n+1}\middle|  \mathcal F_n  \right]
  = E \left[\left(\mu \phi(\theta)\right)^{-(n+1)} \sum_{i=1}^{Z_n} \sum_{k=1}^{Z_1^{(n,i)}} \exp(-\theta T_1^{(n,i)}(k)) \middle| \mathcal F_n \right]
  \\&= \left(\mu \phi(\theta)\right)^{-(n+1)} \sum_{i=1}^{Z_n} \exp(-\theta T_n(i)) E \left[ \sum_{k=1}^{Z_1^{(n,i)}} \exp\left(-\theta\left( T_1^{(n,i)}(k) - T_n(i)\right)\right) \middle| \mathcal F_n \right]
  \\&= \left(\mu \phi(\theta)\right)^{-(n+1)} \sum_{i=1}^{Z_n} \exp(-\theta T_n(i)) E \left[ \sum_{k=1}^{Z_1} \exp\left(-\theta T_1(k)\right) \right]
= Y_n.
\end{align}
This says that $Y_n$ is a non-negative martingale.
Therefore, it has a finite almost sure limit, say $Y_\infty$.
Observe that, we always have
\begin{align}
  \frac{e^{-\theta X_{0,n}}}{\left( \mu \phi(\theta) \right)^n} \leq \frac{1}{\left( \mu \phi(\theta) \right)^n} \sum_{i=1}^{Z_n} \exp\left( -\theta T_n(i) \right).
\end{align}
Therefore, if $\exp(-\theta a)/\mu \phi(\theta) > 1$, then
\begin{align}
P(X_{0,n} \leq an)
&= P(e^{-\theta X_{0,n}} \geq e^{-\theta a n}) 
\leq \frac{E e^{-\theta X_{0,n}}}{e^{-\theta a n}}
\\&\leq \frac{E[ \sum_{i=1}^{Z_n} e^{-\theta T_{n}(i)}]}{\mu^n \phi(\theta)^n} \left(\frac{\mu \phi(\theta)}{e^{-\theta a}}\right)^n
= Y_n (1-\epsilon)^n \to 0.
\end{align}
\end{proof}
% ** Section: HW for 7.1
% *** 7.1.1
\begin{exe}[7.1.1.]
  Given $s< t$ fine $P\left( B(s)>0, B(t)>0 \right)$.
\end{exe}
\begin{proof}
  \begin{align}
&P(B(s)>0, B(t) >0) 
= P(B(s)>0, B(t)- B(s) > - B (s)) 
   \\&= \int_0^\infty dx   \int_{-x}^\infty \frac{1}{\sqrt{2\pi s}}\cdot e^{- \frac{x^2}{2s}}\frac{1}{\sqrt{2\pi (t-s)}} e^{-\frac{y^2}{ 2(t-s)}} dy
\\& = \frac{1}{4} +\frac{1}{2\pi} \arcsin \sqrt{\frac{s}{t}}.
  \end{align}
\end{proof}
% *** 7.1.2
\begin{exe}[7.1.2.]
Find $E(B_1^2 B_2 B_3)$
\end{exe}
\begin{align}
&E [ B_1^2 B_2 B_3] 
  = E\left[ B_1^2 \left(B_1 + \left(B_2 - B_1\right)\right) \left( B_1 + \left( B_2 - B_1 \right) +\left( B_3 - B_2 \right) \right)\right]
  \\&= E[B_1^4] + E\left[ B_1^2 \left( B_2 - B_1 \right)^2\right]
  \\& = E[B_1^4] + E[B_1^2] E\left[ \left( B_2 - B_1 \right)^2 \right]
\\& = 4.
\end{align}
% *** 7.1.4
\begin{exe}[7.1.4.]
$A \in \mathcal F_o$ if and only if there is a sequence of times $t_1, t_2, \cdots$ in $[0,\infty)$ and a $B \in \mathcal R^{\{1,2,\cdots\}}$ so that $A = \{w: (w(t_1), w(t_2),\cdots) \in B\}$.
In words, all events in $\mathcal F_o$ depend on only countably many coordinates.
\end{exe}
\begin{proof}

Let $\Omega = \mathbb R^{[0,\infty)}$. 
Define coordinate process:
\begin{align}
X^{\omega}(t) = \omega(t), \forall \omega \in \Omega, t\geq 0.
\end{align}
Denote by 
\begin{align}
\mathcal I = \{I = (t_k)_{k \in \mathbb N}: \forall k \in \mathbb N, t_k \in [0,\infty)\}
\end{align}
the collection of all the time sequence.
For each time sequence $I=(t_k)_{k \in \mathbb N}\in \mathcal I$, define a map $\psi_I$ from $\Omega$ to $\mathbb R^{\mathbb N}$ such that
\begin{align}
\psi_I(\omega):= (X^{\omega}(t_k))_{k \in \mathbb N}.
\end{align}
Define $[I] := \{t_k: k \in \mathbb N\}$.
Consider a $\sigma$-field on $\Omega$ given by $\mathcal F_{[I]} = \sigma( X(t): t \in [I])$. 
Then by standard measure theory, we have 
\begin{align}
\label{eq:key_for_714}
\mathcal F_{[I]} = \{ \psi_I^{-1} B: B \in \mathcal R^{\mathbb N}\}.
\end{align}

Define the family of subsets of $\Omega$ by 
\begin{align}
\mathcal G := \{A \subset \Omega: \exists I \in \mathcal I, B \in \mathcal R^\mathbb N~s.t.~A = \psi_I^{-1} B\}.
\end{align}
What we needs to proof for this exercise is that $\mathcal G \subset \mathcal F_o$ and $\mathcal F_o \subset \mathcal G$.

1. We claim that $\mathcal G \subset \mathcal F_o$. 
In fact for each $ A \in \mathcal G$, there is an $I \in \mathcal I$ and $B \in \mathcal R^\mathbb N$ such that $A = \psi_I^{-1}B$. 
Therefore $ A \in \mathcal F_{[I]} \subset \mathcal F_o$.

2. We claim that $\mathcal G$ is a $\sigma$-field. 
In fact, if $(A_k)_{k \in \mathbb N}$ is a sequence of elements in $\mathcal G$, then there exists a sequence of $\mathcal I$-elements $(I_k)_{k \in \mathbb N}$ and a sequence of $\mathcal R^\mathbb N$-elements $(B_k)_{k \in \mathbb N}$ such that
\begin{align}
A_k = \psi_{I_k}^{-1}B_k, \quad k \in \mathbb N.
\end{align}
We also have that there exists a $J \in \mathcal I$ such that
\begin{align}
[J] = \bigcup_{k \in \mathbb N} [I_k]
\end{align}
simply because the right hand side is countable. 
Now, from \eqref{eq:key_for_714}, we have $\bigcup_{k\in \mathbb N} A_k \in \sigma(\mathcal F_{[I_k]}: k \in \mathbb N) = \mathcal F_{[J]}\subset \mathcal G$. 
The rest of this claim is elementary.

3. We claim that $\mathcal F_o \subset \mathcal G$. In fact, for each $t\in [0, \infty)$, we have $X(t)$ is $\mathcal G$-measurable simply because $\mathcal F_{t}\subset \mathcal G$. 
\end{proof}
% *** 7.1.5
\begin{exe}[7.1.5.]
Looking at the proof of Theorem 7.1.6. carefully shows that if $\gamma > 5/6$ then $B_t$ is not H\"{o}lder continuous with exponent $\gamma$ at any point in $[0,1]$. 
Show, by considering $k$ increments instead of $3$, that the last conclusion is true for all $\gamma > 1/2 + 1/k$.
\end{exe}
\begin{proof}
Fix a constant $C < \infty$.
Let $A_n = \{w: \exists s \in [0,1]~s.t.~ \forall |t-s|\leq \frac{k}{n}, |B_t - B_s| \leq C |t-s|^\gamma\}$.
For $1 \leq i \leq n - k + 1$, let 
\begin{align}
Y_{i,n} = \max\{|B(\frac{i+j}{n}) - B(\frac{i+j-1}{n})|: j = 0,1,\dots,k - 1\}.
\end{align}
and
\begin{align}
B_n = \{\text{at least one } Y_{k,n} \leq \frac{(2k - 1)C}{ n^{\gamma}}\}.
\end{align}
Then it can be verified that $A_n \subset B_n$.
Therefore
\begin{align}
P(A_n) &\leq P(B_n)
  \leq n P\left( |B(\frac{1}{n})| \leq \frac{(2k - 1) C}{n^{\gamma}} \right)^k
\\& \leq n P(|B(1)|\leq \frac{(2k - 1) C}{n^{\gamma - \frac{1}{2}}})^k
  \\& \leq n \left( \frac{2(2k-1) C}{ \sqrt{2\pi} n^{\gamma - \frac{1}{2}}} \right)^k 
\to 0.
\end{align}
Rest is the same as Theorem 7.1.6.
\end{proof}
% *** 7.1.6
\begin{exe}[7.1.6.]
Fix $t$ and let $\Delta_{m,n} = B(tm2^{-n}) - B(t(m-1)2^{-n})$.
Compute
\begin{align}
  E\left[ \left( \sum_{m \leq 2^n} \Delta^2_{m,n} - t \right)^2 \right]
\end{align}
 and use Borel-Cantelli to conclude that $\sup_{m \leq 2^n} \Delta_{m,n}^2 \to t$ a.s. as $n \to \infty$. 
\end{exe}
\begin{proof}
\begin{align}
  &E\left[\left( \sum_{m=1}^{2^n} \Delta_{m,n}^2 - t \right)^2\right]
    = \sum_{m = 1}^{2^n} E\left( \Delta_{m,n}^2 - \frac{t}{2^n} \right)^2
\\& = 2^n E\left[\left(B(\frac{t}{2^n}) - \frac{t}{2^n}\right)^2\right] = t^2 2^{-n +1}.
\end{align}
Therefore,
\begin{align}
  P\left( \left| \sum_{m=1}^{2^n} \Delta_{m,n}^2 - t \right| > \frac{1}{n} \right)
\leq n^2 t^2 2^{-n +1}.
\end{align}
This says that 
\begin{align}
P\Big( \Big| \sum_{m = 1}^{2^n} \Delta_{m,n}^2 - t \Big| > \frac{1}{n}~i.o.  \Big) = 0.
\end{align}
So we get the desired result by BC Lemma.
\end{proof}
% ** Section: HW for 7.2 and 7.3
% *** 7.2.1.
\begin{exe}[7.2.1.]
  Let $T_0 = \inf\left\{ s> 0: B_s = 0 \right\}$ and let $R = \inf \left\{ t > 1: B_t = 0 \right\}$. $R$ is for right or return. Use the Markov property at time $1$ to get 
  \begin{align}
P_x(R > 1+t)  = \int p_1(x,y) P_y(T_0 > t) dy
  \end{align}
\end{exe}
\begin{proof}
Notice that $R = T_0 \circ \theta_1 + 1$, therefore from Theorem 7.2.1. we have
\begin{align}
  &E_x\left(  \mathbf 1_{R(\cdot)> 1+t} \middle| \mathcal F_1^+ \right)
= E_x\left(  \mathbf 1_{ (T_0 \circ \theta_1)>t} \middle| \mathcal F_1^+ \right)
\\&  = E_x\left[  \left(\mathbf 1_{ T_0>t} \circ \theta_1 \right)(\omega) \middle| \mathcal F_1^+ \right]
  \\&  = E_{B_1}\left[  \mathbf 1_{ T_0>t} \right] = \int p_1(x,y) P_y(T_0 > t) dy.
\end{align}
\end{proof}
% *** 7.2.3.
\begin{exe}[7.2.3.]
Let $a< b$, then with probability one $a$ is the limit of local maximum of $B_t$ in $(a,b)$.
So the set of local maxima of $B_t$ is almost surely a dense set.
However, unlike the zero set it is countable.
\end{exe}
\begin{proof}
From Theorem 7.2.5., we know that $T_0:= \inf\{t \in (a,b): B_t = B_a\} = a$ almost surely. This says that there exists a $\Omega_0$ with probability $1$ such that for any $\omega \in \Omega_0$, there exists a strictly decreasing sequence of $t_n$ with $B_{t_n} = B_a$ and $t_n \downarrow a$.

From Theorem 7.2.4., we know that $T_1:= \inf\{t \in (a,b): B_t > B_a\} = a$ almost surely. This says that there exists a $\Omega_1$ with probability $1$ such that for any $\omega \in \Omega_1$, there exists a strictly decreasing sequence of $s_n$ with $B_{s_n} > B_a$ and $s_n \downarrow a$.

Therefore, by chosen suitable subsequence, we know that for each $\omega \in \Omega_0 \cap \Omega_1$, there exists a strictly decreasing sequence
\begin{align}
t_1 > s_1 > t_2 > s_2 > \dots 
\end{align}
such that $B_{t_k} = B_a$ and $B_{s_k} > B_a$ for each $k \in \mathbb N$ and that both $t_k \downarrow a$ and $s_k \downarrow a$ hold.
Now since the Brownian path are continuous, there will be a sequence of $(r_k)_{k\in \mathbb N}$ such that
\begin{align}
t_1 > r_1 > t_2 > r_2 > \dots
\end{align}
and each $r_k$ is a local maxima. (Simply chose $r_k\in (t_k, t_{k+1})$ such that $B_{r_k} = \max\{ B_r: r\in [t_k, t_{k+1}] \}$.)

To summarize, for each $a< b$, we have almost surely that $a$ is the limit of local maximum of $B_t$ in $(a,b)$.
Therefore, almost surely, for each $q\in \mathbb Q$, we have $q$ is in the closure of the set of local maximum of the Brownian path. 
In another word, almost surely, the set of local maxima of Brownian path is a dense set.
\end{proof}
% *** 7.2.4.
\begin{exe}[7.2.4.]
(i) Suppose $f(t)>0$ for all $t>0$. 
Use Theorem $7.2.3.$ to conclude that $\limsup_{t\downarrow 0}B(t)/f(t) = c$, $P_0$ a.s., where $c \in [0,\infty]$ is a constant.
(ii) Show that if $f(t) = \sqrt{t}$ then $c = \infty$, so with probability one Brownian paths are not H\"older continuous of order $1/2$ at $0$. 
\end{exe}
\begin{proof}
(i). Define $C = \limsup_{t\downarrow 0} B(t)/f(t)$, then $C$ is a random variable which is $\mathcal F_0^+$-measurable. It can also be verified that $C$ takes values in $[0, \infty]$ from Theorem 7.2.4., since there exists a sequence of strictly decreasing $t_0 \downarrow 0$ with $B(t_0)>0$.

Now use Theorem 7.2.3. we know that $C$ almost surely is a constant.

(ii). Define $X_t = t B(1/t)$, then by Theorem 7.2.6. and Theorem 7.2.8 we have
\begin{align}
&\limsup_{t\downarrow 0} \frac{B(t)}{ \sqrt{t}} 
  = \limsup_{t\downarrow 0} \frac{t X(1/t)}{\sqrt{t}}
\\&= \limsup_{u\to \infty} \frac{X(u)}{\sqrt{u}}
= \infty.
\end{align}
\end{proof}
% *** 7.3.1.
\begin{exe}[7.3.1.]
Let $A$ be an $F_\sigma$, that is, a countable union of closed sets.
Show that $T_A = \inf\left\{ t : B_t \in A \right\}$ is a stopping time.
\end{exe}
\begin{proof}
Let 
\begin{align}
A = \bigcup_{n \in \mathbb N} K_n
\end{align}
where $K_n$ are closed sets. 
Define closed sets $A_n = \bigcup_{k = 1}^n K_k$, then we have
\begin{align}
A = \bigcup_{n \in \mathbb N} A_n
\end{align}
Define $T_{(A_n)} = \inf\{t : B_t \in A_n\}$ which by Theorem 7.3.4. are stopping times.
Now notice that 
\begin{align}
\bigcup_{n \in \mathbb N} \{t : B_t \in A_n\} = \{t: B_t \in A\}.
\end{align}
So we must have 
\begin{align}
T_A = \inf_{n\in \mathbb N} T_{(A_n)} = \lim_{n\to \infty} T_{(A_n)}. 
\end{align}
Theorem 7.3.2. then says that $T_A$ is a stopping time.
\end{proof}
% *** 7.3.2.
\begin{exe}[7.3.2.]
If $S$ and $T$ are stopping times, then $S \wedge T = \min \{S, T\}$, $S\vee T = \max\{S, T\}$, and $S+T$ are also stopping times.
In particular, if $t \geq 0$, then $S \wedge t, S \vee t$ and $S+ t$ are stopping times.
\end{exe}
\begin{proof}
It can be verified that for each $t> 0$,
\begin{align}
\{S \wedge T > t\} = \{S > t\} \cap \{T > t\} \in \mathcal F_t,
\end{align}
and
\begin{align}
\{S \vee T > t\}= \{S>t\} \cup \{T > t\} \in \mathcal F_t,
\end{align}
and
\begin{align}
&\{S + T < t\} = \{\exists q_1,q_2 \in \mathbb Q,~\text{s.t.}~S \leq q_1, T \leq q_2, q_1 + q_2 < t\}
\\&= \bigcup_{q_1,q_2 \in \mathbb Q: 0 \leq q_1 + q_2 \leq t} \{S \leq q_1, T\leq q_2\} \in \mathcal F_t. 
\end{align}
\end{proof}
% *** 7.3.4.
\begin{exe}[7.3.4.]
Let $S$ be a stopping time, let $A \in \mathcal F_S$, and let $R = S$ on $A$ and $R = \infty$ on $A^c$. 
Show that $R$ is a stopping time.
\end{exe}
\begin{proof}
It can be verified that for each $t\in [0,\infty)$
  \begin{align}
\{R \leq t\} = \{A, S \leq t\} \in \mathcal F_t.
  \end{align}
\end{proof}
% *** 7.3.5.
\begin{exe}[7.3.5.]
Let $S$ and $T$ be stopping times. 
$\{S < T\}$, $\{S > T\}$, and $\{ S = T\}$ are in $\mathcal F_S$ (and in $\mathcal F_T$).
\end{exe}
\begin{proof}
It can be verified that, for each $s> 0$,
\begin{align}
&\{S < T\} \cap \{S < s\} = \{\exists q \in \mathbb Q \cap (0,s),~ \text{s.t.}~S < q < T \}
\\&= \bigcup_{q\in \mathbb Q \cap(0,s)} \{ S < q, T>q\} 
\in \mathcal F_s.
\end{align}
This shows that $\{S < T\}\in \mathcal F_S$.
It can also be verified that, for each $s>0$,
\begin{align}
&\{S > T\} \cap \{S < s\} 
= \{\exists q \in \mathbb Q,~ \text{s.t.}~T < q < S < s\}
\\&= \bigcup_{q\in \mathbb Q:0 < q < s}\left(\{ T < q \} \cup \{q<S<s\}\right) 
\in \mathcal F_s. 
\end{align}
This says that $\{S> T\}\in \mathcal F_S$. Finally
\begin{align}
  \{S = T\} = \left( \{S>T\} \cup \{S< T\} \right)^c \in \mathcal F_S.
\end{align}
\end{proof}
% *** 7.4.1
\begin{exe}
Let $B_t = (B_t^1, B_t^2)$ be a two-dimensional Brownian motion starting from $0$.
Let $T_a = \inf\{t: B_t^2 = a\}$.
Show that (1) $B^1(T_a), a\geq 0$ has independent (stationary) increments and (2) $B^1(T_a) = _d aB^1(T_1)$.
Use this to conclude that $B^1(T_a)$ has a Cauchy distribution. 
\end{exe}
\begin{proof}
  (1) Suppose that $0 = a_0 < a_1 < \dots a_{n} = u < a_{n+1} = v$. Thanks to induction, in order to show that increments $\{B^1(T_{(a_{(k+1)})}) - B^1(T_{(a_k)}):k=0,\dots,n\}$ are mutually independent with each other, we only have to show that the last increment $B^1(T_{v})- B^1(T_{u})$ is independent of the previous increments $\{B^1(T_{(a_{(k+1)})}) - B^1(T_{(a_k)}):k=0,\dots,n - 1\}$.
Notice that all $(T_{a})_{a\geq 0}$ are stopping times of the two-dimensional Brownian motion $(B_t)_{t\geq 0}$ whose corresponding filtration will be denote by $(\mathcal F_t)_{t\geq 0}$. 
Then it is easy to see that the previous increments $\{B^1(T_{(a_{k+1})}) - B^1(T_{(a_k)}):k=0,\dots,n - 1\}$ are $\mathcal F_{(T_{u})}$-measurable.
So we only have to show that $B^1(T_v) - B^1(T_u)$ is independent of the $\sigma$-field $\mathcal F_{(T_u)}$. 
Now from the strong Markov property: for any bounded function $f$,
\begin{align}
&E\left[f\left(B^1(T_v) - B^1(T_u)\right) | \mathcal F_{(T_u)}\right]
= E_{(B^1(T_u), u)} \left[ f\left( B^1(T_v) - B^1(0) \right) \right]
\\&= E_{(0,0)}\left[ f \left( B^1(T_{v-u}) \right)\right].
\end{align}
The RHS is non-random, so $B^1(T_v)- B^1(T_u)$ is independent of the $\sigma$-field $\mathcal F_{(T_u)}$.
The RHS also says that the distribution of $B^1(T_u) - B^1(T_v)$ is only dependent on $v-u$. This is the stationary part.

(2) Note that from (7.1.1.) we have for each $a > 0$
\begin{align}
(X_s)_{s\geq 0}:=(a^{-1}B_{a^2s})_{s\geq 0} \overset{d}{=} (B_s)_{s\geq 0}
\end{align}
Therefore,
\begin{align}
B^1(T_a) 
&= B^1(\inf\{t: B^2(t) = a\}) 
= B^1(a^2\inf\{ s: a^{-1}B^2(a^2 s) = 1\})
\\&= a \cdot X^1(\inf\{s:X^2_s = 1\})
\overset{d}{=}aB^1(T_1)  
\end{align}

(3). From (1) we have that
\begin{align}
B^1(T_n) \overset{d}= \sum_{k=0}^{n} Y_k
\end{align}
where $(Y_k)_{k\in \mathbb Z_+} \overset{d}= \left( B(T_{k+1}) - B(T_k) \right)_{k \in \mathbb Z_+}$ are i.i.d. random variables. So now, we have that
\begin{align}
B^1(T_1) \overset{d}{=} n^{-1}B^1(T_n)\overset{d}{=}n^{-1}\sum_{k=1}^n Y_k 
\xrightarrow[n\to \infty]{} B^1(T_1).
\end{align}
Simply notice that $(B^1_t,B^2_t)_{t\geq 0} \overset{d}= (-B^1_t,B^2_t)$, one can verify that $(Y_k)_{k\in \mathbb Z_+}$ actually have symmetric distribution.
So $B^1(T_1)$ must have symmetric $\alpha$-stable distribution with $\alpha = 1$, which by its definition, is Cauchy distribution.
\end{proof}
% *** 7.4.2
\begin{exe}[7.4.2.]
Use (7.2.3) to show that $R:= \inf\{t>1: B_t = 0\}$ has probability density
\begin{align}
P_0(R -1\in dt)
= \frac{1}{(\pi t^{1/2}(1+t))} dt.
\end{align}
\end{exe}
\begin{proof}
According to (7.2.3.) and (7.4.6), we have
\begin{align}
&P_0(R-1 \in dt) = \int_{y\in \mathbb R} p_1(0,y) P_y(T_0 \in dt) dy
\\&= \int_{y\in \mathbb R} p_1(0,y) P_0(T_y \in dt) dy
= \int_{y\in \mathbb R} p_1(0,y) \frac{1}{\sqrt{2\pi s^3}} y e^{- \frac{y^2}{2t}}~dt~dy
\\&= \frac{1}{\pi t^{1/2} (1+t)} dt.
\end{align}
\end{proof}
% *** 7.4.4
\begin{exe}[7.4.4.]
Let $A_{s,t}$ be the event Brownian motion has at least one zero in $[s,t]$. 
Show that $P_0(A_{s,t}) = \frac{2}{\pi} \arccos(\sqrt{s/t})$.
\end{exe}
\begin{proof}
According to 7.4.6., note that 
  \begin{align}
&P_0(A_{s,t}) 
= 2\int_0^\infty p_s(0,x) P_x(T_0 \leq t-s) dx
    \\&=2\int_0^\infty (2\pi s)^{-1/2s} e^{- x^2/2} \int_0^{t-s} (2\pi u^3)^{-1/2} x e^{-x^2/2u}~du~dx
\\&= \frac{2}{\pi} arccos(\sqrt{s/t}).
  \end{align}
\end{proof}
% *** 7.5.1
\begin{exe}[7.5.1.]
  Let $T = \inf\{B_t \not\in (-a,a)\}.$ Show that
  \begin{align}
    E_0 \exp(-\lambda T) 
    = 1/ \cosh(a \sqrt{2\lambda})
\end{align}
\end{exe}
\begin{proof}
According to the fact that $e^{\theta B_t - \frac{1}{2} \theta^2 t}$ is a martingale, and the fact (from symmetry) that conditionally given $T$, $B_T = a$ or $B_T = -a$ with $1/2$ probability, we have that
\begin{align}
  1 &= E_0\left[ e^{\theta B_T - \frac{1}{2} \theta^2 T} \right]
  \\&= E_0\left[ e^{-\frac{1}{2}\theta^2 T} \right] \cdot \frac{e^{\theta a} + e^{-\theta a}}{2}.
\end{align}
This implies the desired result.
\end{proof}
% *** 7.5.3
\begin{exe}[7.5.3]
Let $\sigma = \inf\{t: B_t \not\in (a,b)\}$ and let $\lambda > 0$.

(1) Use the strong Markov property to show
\begin{align}
E_x \exp{-\lambda T_a} 
  = E_x \left( e^{-\lambda \sigma}; T_a < T_b \right) + E_x\left( e^{-\lambda \sigma}; T_b < T_a \right) E_b \exp\{- \lambda T_a\}. 
\end{align}

(2) Interchange the roles of $a$ and $b$ to get a second equation, use Theorem 7.5.7. to get
\begin{align}
  E_x\left[ e^{-\lambda \sigma}; T_a < T_b \right]
 = \sinh(\sqrt{2\lambda} (b-x))/\sinh(\sqrt{2\lambda}(b-a)).
\end{align}
\end{exe}
\begin{proof}
(1). We have that $E_x[e^{- \lambda (T_a - \sigma)}| \mathcal F_\sigma] = E_{B_\sigma}[e^{-\lambda T_a}]$. Therefore,
\begin{align}
  &E_x e^{-\lambda T_a} = E_x \left[ e^{-\lambda \sigma} E\left[ e^{-\lambda (T_a - \sigma)} \middle| \mathcal F_\sigma\right]\right]
  \\&= E_x\left[ e^{-\lambda \sigma}, T_a < T_b \right] + E_x[e^{-\lambda \sigma},T_a>T_b] E_b e^{-\lambda T_a}.
\end{align}

(2). Change $a$ and $b$ in the above, we have
\begin{align}
  E_x\left[ e^{-\lambda T_b} \right]
  = E_x\left[ e^{-\lambda \sigma}; T_b < T_a \right] + E_x\left[ e^{-\lambda \sigma}; T_a < T_b \right] E_a[e^{-\lambda T_b}].
\end{align}
According to Theorem 7.5.7., we have 
\begin{align}
  E_x\left[ e^{-\lambda T_a} \right] &= \exp(-(a-x)\sqrt{2\lambda});
  \\E_x\left[ e^{-\lambda T_b} \right] &= \exp(-(x-b)\sqrt{2\lambda});
\\ E_a[e^{-\lambda T_b}] &= E_b[e^{-\lambda T_a}] = \exp(-(b-a)\sqrt{2\lambda}).
\end{align}
From those, we can obtain the desired result by solving a linear equation.
\end{proof}
% *** 7.5.5
\begin{exe}[7.5.5.]
Find a martingale of the form $B_t^6 - c_1tB_t^4+c_2t^2B_t^2-c_3t^3$ and use it to compute the third moment of $T = \inf\{t: B_t \neq (-a,a)\}$.
\end{exe}
\begin{proof}
It can be verified that when $c_1 = 15, c_2 = 45, c_3 = 15$, function 
\begin{align}
u(x,t):= x^6 - c_1 tx^4 + c_2 t^2 x^2 - c_3 t^3
\end{align}
solves equations
\begin{align}
\frac{\partial u}{\partial t} + \frac{1}{2} \frac{\partial^2 u}{ \partial x^2} = 0.
\end{align}
Therefore $B_t^6 - c_1tB_t^4+c_2t^2B_t^2-c_3t^3$ is a martingale.
Now it can be calculated from bounded convergence and monotone convergence that
\begin{align}
  E\left[ T^3 \right]
&= \lim_{t\to \infty} E\left[(T\wedge t)^3\right] 
  \\& =\lim_{t\to \infty} c_3^{-1}E\left[ B_{T\wedge t}^6 - c_1(T\wedge t)B_{T\wedge t}^4+c_2(T\wedge t)^2 B_{T\wedge t}^2 \right]
  \\& = c_3^{-1}E\left[ a^6 - c_1Ta^4+c_2T^2a^2 \right]
= \frac{61}{15}a^6.
\end{align}
\end{proof}
% *** 7.6.2.
\begin{exe}[7.6.5.]
Show that $B_t^3 - \int_0^t 3B_s ds$ is a martingale.
\end{exe}
\begin{proof}
According to Ito's formula,
\begin{align}
 d\left( B_t^3 - \int_0^t 3 B_s ds \right)
= 3B_t^2 dB_t + \frac{1}{2} \cdot 6 B_t dt - 3 B_t dt
= 3 B_t^2 dB_t.
\end{align}
This and Theorem 7.6.4. imply that $B_t^3 - \int_0^t 3B_sds$ is a martingale.
\end{proof}
% *** 7.6.3.
\begin{exe}[7.6.3.]
Let $\beta_{2k}(t) = E_0 B_t^{2k}$. 
Use Ito's formula to relate $\beta_{2k}(t)$ to $\beta_{2k-2}(t)$ and use this relationship to derive a formula for $\beta_{2k}(t)$.
\end{exe}
\begin{proof}
According to Ito's formula:
\begin{align}
dB_t^{2k} = 2k B_t^{2k - 1} dB_t + \frac{1}{2} 2k(2k-1) B_t^{2k-2} dt.
\end{align}
Therefore,
\begin{align}
  &\beta_{2k}(t) 
= E\left[ B_t^{2k} \right] 
   = E\left[ \int_0^t 2k B_s^{2k - 1} dB_s + \frac{2k(2k-1)}{2} \int_0^t B_s^{2k-2} ds \right]
  \\& = \frac{2k(2k-1)}{2} \int_0^t E\left[  B_s^{2k-2}  \right] ds
  = \frac{2k(2k-1)}{2} \int_0^t E\left[  B_s^{2k-2}  \right] ds
  \\&= \frac{2k(2k-1)}{2} \int_0^t \beta_{2(k-1)}(s) ds.
\end{align}
Now, it is trivial to verify that 
\begin{align}
\beta_{2k}(t) = (2k-1)!!\cdot t^{k}.
\end{align} 
\end{proof}
% *** 7.6.5.
\begin{exe}[7.6.5.]
Apply Ito's formula to ($d$-dimensional) $|B_t|^2$. 
Use this to conclude that $E_0 |B_t|^2 = td$.
\end{exe}
\begin{proof}
According to Ito's formula Theorem 7.6.7., let $f(x) = |x|^2, x\in \mathbb R^d$, we have 
\begin{align}
  &df(B_t)
= \sum_{k=1}^d D_kf(B_s) dB^{(k)}_s + \frac{1}{2} \sum_{k=1}^d D_{kk} f(B_s)ds
\\&  = \sum_{k=1}^d 2B^{(k)}_s dB^{(k)}_s + \sum_{k=1}^d ds
\end{align}
Therefore, thanks to Theorem 7.6.4.,
\begin{align}
  E_0\left[ |B_t|^2 \right] 
= E_0\left[ 2\sum_{k=1}^d \int_0^t B^{(k)}_s dB^{(k)}_s + \sum_{k=1}^d\int_0^t ds \right]
= td.
\end{align}
\end{proof}
% *** 8.1.1.
\begin{exe}[8.1.1.]
(In the context of the proof of Theorem 8.1.1.,) use Exercise 7.5.4. to conclude that $E(T_{U,V}^2)\leq 4 EX^4$.
\end{exe} 
\begin{proof}
Recall the Exercise 7.5.4. which says that if $T_{a,b} = \inf\{t: B_t\not\in (a,b)\}$ where $a < 0< b$, then $E T^2 \leq 4 E(B_T^4)$ and $EB_T^4 \leq 36 ET^2$. Now, noting that for each $a<0<b$ $T_{a,b}, B_{(T_{a,b})} \in \mathcal F^B$ is independent of $U$ and $V$, and that from Theorem 8.1.1 $B_{(T_{U,V})} \overset{d}= X$, we have
\begin{align}
 & E\left[ T_{U,V}^2 \right] 
  = \int  E_0\left[ T^2_{u,v} \right]  P\left( U\in du, V \in dv \right) 
  \\& \leq 4 \int E_0\left[ B_{T_{u,v}}^4 \right] P\left( U \in du, V \in dv \right)  
  = 4 E_0 \left[ B_{(T_{U,V})}^4 \right] 
= 4 E X^4 .
\end{align} 
\end{proof}
% *** 8.1.2.
\begin{exe}[8.1.2.]
Suppose $S_n$ is one-dimensional simple random walk and let 
\begin{align}
R_n = 1 + \max_{m\leq n} S_m - \min_{m\leq n} S_m
\end{align}
be the number of points visited by time $n$. 
Show that 
\begin{align}
R_n / \sqrt{n} \xrightarrow[t\to \infty]{d} \text{something}.
\end{align}
\end{exe}
\begin{proof}
For each continuous function $f$ on $[0,1]$, consider functional 
$Rf:= \sup_{t\in [0,1]} f(t) - \inf_{t\in [0,1]} f(t) $.
Then $R$ is a continuous map from $C[0,1]$ to $\mathbb R$.
Now Theorem 8.1.5. says that
\begin{align}
R \left( S(n \cdot)/\sqrt{n} \right) \xrightarrow[n\to \infty]{d} R B,
\end{align}
where $B$ is a Brownian motion on $[0,1]$, and therefore $RB$ is a random variable.
Finally, note that
\begin{align}
(R_n-1)/\sqrt{n} = R(S(n\cdot)/\sqrt{n}), \quad n\geq 0.
\end{align}
\end{proof}
% ** END
\end{document}
