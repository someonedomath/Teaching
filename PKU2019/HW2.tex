% * Preamble
% ** Document class
\documentclass[12pt,a4paper]{amsart}
\setlength{\textwidth}{\paperwidth}
\addtolength{\textwidth}{-2in}
\calclayout
% ** Encoding information
\usepackage[utf8]{inputenc}
\usepackage[T1]{fontenc}
% \usepackage[UTF8]{ctex} 
% You need XeTeX to build ctex based document 
% ** packages about math
\usepackage{mathtools}
\mathtoolsset{showonlyrefs}
\usepackage{mathrsfs}
\usepackage{stackrel}
\usepackage[backref]{hyperref}
\usepackage{comment}
% ** theorem environment definition
\usepackage{amsthm}
\theoremstyle{plain}
\newtheorem{thm}{Theorem}[section]
\newtheorem{lem}[thm]{Lemma}
\newtheorem{prop}[thm]{Proposition}
\newtheorem{cor}[thm]{Corollaray}
\newtheorem{conj}[thm]{Conjecture}
\theoremstyle{definition}
\newtheorem{defi}[thm]{Definition}
\newtheorem{exa}[thm]{Example}
\newtheorem{asp}{Assumption}
\theoremstyle{remark}
\newtheorem*{exe}{Exercise}
\newtheorem{rem}[thm]{Remark}
\numberwithin{equation}{section}
\allowdisplaybreaks
% * Document
% ** BEGIN
\begin{document}
% ** Basic Info
\title
[short title]
{long title}
\author
[Z. Sun]
{Zhenyao Sun}
\address
{Zhenyao Sun\\
School of Mathematical Sciences\\
Peking University\\
Beijing, P. R. China, 100871}
\email{zhenyao.sun@pku.edu.cn}
\begin{abstract}
  TBD
\end{abstract}
\maketitle
% ** Section: HW for 6.5
\begin{exe}[6.5.1]
Consider the convergence in (a) of Theorem 6.4.1. Let $X_{m,m+k}=f(k)\geq 0$, where $f(k)/k$ is decreasing, and check that $X_{m,m+k}$ is subaddditive.
\end{exe}
\begin{exe}[6.5.2.]
Consider the longest common subsequence problem, Example 6.4.4. when $X_1,X_2,\dots$ and $Y_1,Y_2,\dots$ are i.i.d. and take the values $0$ and $1$ with probability $1/2$ each.
(a) Compute $EL_1$ and $EL_2/2$ to get lower bounds on $\gamma$.
(b) Show $\gamma < 1$ by computing the expected number of $i$ and $j$ sequence of length $K=an$ with the desired property. 
\end{exe}
\begin{exe}[6.5.3.]
Given a rate one Poisson process in $[0,\infty)\times [0,\infty)$, let $X_1,Y_1$ be the point that minimizes $x+y$.
Let $(X_2,Y_2)$ be the point in $[X_1,\infty) \times [X_1,\infty)$ that minimizes $x+y$, and so on.
Use this construction to show that in Example 6.5.2. $\gamma \geq (8/\pi)^{1/2} > 1.59$. 
\end{exe} 
\begin{exe}[6.5.4.]
Let $\pi_n$ be a random permutation of $\{1,\dots,n\}$ and let $J_k^n$ be the number of subsets of $\{1,\dots,n\}$ of size $k$ so that the associated $\pi_n(j)$ form an increasing subsequence.
Compute $EJ_k^n$ and take $k\sim \alpha n^{1/2}$ to conclude that in Example 6.5.2. $\gamma \leq e$.
\end{exe}
\begin{exe}[6.5.5.]
Let $\phi(\theta) = E \exp(-\theta t_i)$ and
\begin{align}
Y_n =\left( \mu \phi(\theta) \right)^{-n} \sum_{i=1}^{Z_n} \exp\left( -\theta T_n(i) \right)
\end{align}
where the sum is over individuals in generation $n$ and $T_n(i)$ is the $i$th person's birth time.
Show that $Y_n$ is a nonnegative martingale and use this to conclude that if $\exp(-\theta a)/\mu \phi(\theta) > 1$, then $P(X_{0,n}\leq an)\to 0$. 
A little thought reveals that this bound is the same as the answer in the answer in the last exercise.
\end{exe}
% ** END
\end{document}